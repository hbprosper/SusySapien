%$Revision:$
\documentclass[aps,prd,preprint]{revtex4}
\usepackage{graphicx,amssymb,amsmath}
\begin{document}

\centerline{\bf \large Razor Boost Without Explicit Fitting}

\centerline{Sezen Sekmen$^1$ Nadja Strobbe$^2$ and Harrison B. Prosper$^3$}

\centerline{$^1$ CERN, Geneva, Switzerland}
\centerline{$^2$ Ghent University, Ghent, Belgium}
\centerline{$^3$ Florida State University, Tallahassee, USA}

\centerline{CMS Razor Group}
\centerline{\today}

\bigskip

\bigskip

\noindent
{\bf \large Razor Analyses}

A key assumption of the razor analyses, which are based on the razor variables $M_R$ and $R^2$, is that the background density in the razor plane ($M_R$, $R^2$) can be adequately modeled using the
4-parameter density
\begin{eqnarray}
p(x, y|\theta) & = & (z - 1) e^{-n \, z}, \nonumber \\
	\text{where}\quad z 	& = & b \, |(x - x_0) \, (y - y_0)|^{1/n}, \nonumber \\
				x	& = & M_R, \nonumber \\
				y	& = & R^2, 
				\label{eq:pxy}
\end{eqnarray}
and $\theta = x_0, y_0, b, $ and $n$ are free parameters. The razor likelihood in
either the side-band region or the signal region is given by a
marked Poisson process
\begin{equation}
	L(x, y|\mu, \theta) = \text{Poisson}(N|\lambda) \, \prod_{j=1}^K \left( s \, p_s(x_j, y_j|\mu) + \sum_{l=1}^M b_l \, p_{bl}(x_j, y_j|\theta_l) \right),
\end{equation}
where $N$ is the total observed count, $n_s$ and $b_l$ are the expected signal and background counts, respectively, $\lambda = n_s + \sum_{l=1}^M b_l$ is the total expected count, and
$p_s$ and $p_{bl}$ are the \emph{normalized} signal and background
densities, respectively. In general, the background parameters $\theta_l$ differ from one background component to another.

In the current razor analyses, a maximum likelihood fit is performed in a side-band region
in which it is assumed that the signal content is negligible. The fit provides estimates 
$\hat{x}_0, \hat{y}_0, \hat{b}, \hat{n}$ and the associated $4 \times 4$ covariance matrix. 
In effect, the likelihood in the side-band region is approximated by a 4-variate Gaussian.
The
model in Eq.~(\ref{eq:pxy}) is then applied to the signal region, along with the
multivariate Gaussian, which serves as a prior that constrains the nuisance
parameters $x_0, y_0, b, n$ in the signal region. 

The advantage of this approach is that a background model can be constructed once and for
all and re-used with different signal hypotheses. The disadvantages are the need to assume
negligible signal contamination in the side-band and the neglect of possible non-Gaussian tails in the side-band likelihood.

\bigskip

\noindent
{\bf \large Alternative to Fitting}
It is always helpful to remind ourselves of the experimental goal. Our goal
is to measure
the effective differential cross section $d^2\sigma_\text{eff}/dx dy$ across the razor plane, where
$\sigma_\text{eff} = \epsilon \, \sigma \text{BR}$ in which $\epsilon$ is the signal acceptance and
$\sigma \text{BR}$ is the cross section times branching ratio. 
Given this experimental information, together with a clear, precise, and complete description of
the event selection and a readily accessible approximation of the CMS detector response, any 
theoretical model that yields predictions for the differential cross section in the razor plane can 
be tested.

A straightforward way to achieve the goal is to discretize the razor plane and measure the signal
content of each bin. This reduces the analysis problem to the well-understood multi-count
experiment for which the likelihood is generally taken to be
\begin{equation}
	L(x, y|\sigma, b, {\cal L}) = \prod_{j=1}^K \text{Poisson}(N_j| \sigma_j \, {\cal L} +  b_\text{tot} \,\sum_{l=1}^M b_{lj} / b_\text{norm}),
\end{equation}
where, now, $K$ is the number of bins in the razor plane, $N_j$ is the observed count
in bin $j$, $\sigma$ denotes the effective cross sections $\sigma_1,\ldots,\sigma_K$, 
\begin{eqnarray}
	b(\theta) & = & \int_{x_\text{min}}^{x_\text{max}}  dx \int_{x_\text{min}}^{x_\text{max}} dy  \, p(x, y|\theta), \quad \text{and} \\ \nonumber
	b_\text{norm} & = & \sum_{j=1}^K \sum_{l=1}^M b_{lj}.
\end{eqnarray}
The quantity $b_\text{tot} (b_{lj} / b_\text{norm})$ is the expected background count in a given bin for a given background component,
where $b_\text{tot}$ is the total expected background in the razor plane. The likelihood can
then be reduced to a function of the effective cross sections only, either by profiling over the 
background parameters $\theta_l$ and the integrated luminosity ${\cal L}$, constrained 
by the luminosity prior $\pi({\cal L})$, or by marginalization. This model is one that could be implemented using {\tt HistFactory} at a later date.
%
%\begin{thebibliography}{99}
%\bibitem{bib:Gao}
%J. Gao et al., ``Next-to-leading QCD effect to the quark compositeness search at the LHC",
%\emph{Phys. Rev. Lett.} {\bf 106} (2011) 142001, {\tt arXiv:1101.4611}.
%\end{thebibliography}

\end{document}